% Inicio de \section*{Introducción}\label{introduccion}
\chapter*{Introducción}\label{introduccion}

% [Redacte de manera coherente en una cuartilla cuál es la nueva contribución, su importancia y por qué es adecuado para sistemas computacionales. Se sugiere para su redacción seguir los cinco pasos siguientes: 1) Establezca el campo de investigación al que pertenece el proyecto, 2) describa los aspectos del problema que ya han sido estudiado por otros investigadores, 3) explique el área de oportunidad que pretende cubrir el proyecto propuesto, 4) describa el producto obtenido y 5) proporcione el valor positivo de proyecto.]

A lo largo de la historia del área de computación muchos profesores han buscado formas de solucionar la complejidad de enseñar programación. Entre estos, en 1970 Pascal que busco eliminar la variación de entornos de programación, los programas requerían cambios según las características técnicas de la máquina que lo ejecuta. Para menores de edad se han creado a lo largo de los años programas como LOGO, Scratch, Lego WeDo o iniciativas como Code.org que han buscado hacer interesante el proceso de aprendizaje con entornos gráficos dinámicos para mantener captada la atención de los estudiantes. Se han creado proyectos similares para enseñar lenguajes de programación como lo son CodingGames o CodeCombat por nombrar algunos y usan lenguajes como \textit{Javascript} o \textit{Python}. Sin embargo, no hay mucho material que cubra pseudocódigo.

Es importar notar que aprender a programar es una labor compleja donde se tienen que atacar distintos problemas al mismo tiempo. Programadores por primera vez tienen que aprender la sintaxis del lenguaje de programación de elección, así como sus idiosincrasias. Estos programadores a su vez tienen que aprender distintas abstracciones, como estructuras de datos para modelar problemas. Adicionalmente, tienen que aprender estrategias para la creación de algoritmos, como descomposición de problemas y reconocer patrones para buscar similitudes entre problemas.

En este trabajo se creó un videojuego multijugador. Todos los alumnos de una clase pueden entrar al mismo tiempo y jugar entre ellos. En este juego los estudiantes resuelven acertijos con pseudocódigo: tienen que analizar que tarea realiza un programa dadas ciertas entradas, completar programas para que realicen una tarea dada, de forma que trabajen su modelo mental sobre los distintos comandos vistos en el curso y trabajan en sus habilidades para crear algoritmos. Incluye los temas de: variables, estructuras secuenciales y ciclos, estos detallados en la retícula del curso de Fundamentos de Programación de la Universidad Autónoma de ciudad Juárez. En este videojuego, se divide a los jugadores en dos grupos: los programadores y los impostores. Los programadores, su principal objetivo es resolver todos los acertijos; los impostores tienen que impedir que los programadores terminen sus tareas, pueden hacerlo matando en juego a los programadores, o distrayéndolos, adicionalmente, pueden resolver acertijos para obtener extras que les de ventaja. 

Para ver su utilidad, se hizo entrevistas con profesores donde se veía el material del juego, a fin de que comentaran debilidades y ventajas que encontraron. Y en general, los profesores mostraron interés en el juego y piensan que una vez publicado es algo que a sus alumnos les llamara la atención. Algunos de estos docentes comentaron que es algo que usarían en el salón de clases. En estas reuniones se encontraron mejoras a realizar a futuro: como temas adicionales, mayor complejidad y algunos cambios a los \textit{puzzles} para ser más provechos a los estudiantes y a la vez no tengan duda de las instrucciones.
