% Inicio de \section*{Introducción}\label{introduccion}
\chapter*{Introducción}\label{introduccion}

% [Redacte de manera coherente en una cuartilla cuál es la nueva contribución, su importancia y por qué es adecuado para sistemas computacionales. Se sugiere para su redacción seguir los cinco pasos siguientes: 1) Establezca el campo de investigación al que pertenece el proyecto, 2) describa los aspectos del problema que ya han sido estudiado por otros investigadores, 3) explique el área de oportunidad que pretende cubrir el proyecto propuesto, 4) describa el producto obtenido y 5) proporcione el valor positivo de proyecto.]

A lo largo de la historia del área de computación muchas personas han buscado formas de solucionar la complejidad de enseñar programación. Se han creado para menores de edad programas como \textit{LOGO}, \textit{Scratch}, el kit \textit{Lego WeDo} o iniciativas como \textit{Code.org} que han buscado hacer interesante el proceso de aprendizaje con entornos gráficos dinámicos para mantener captada la atención de los estudiantes mientras aprenden a programar, normalmente usando entornos de programación como basados en bloques. A su vez, se han creado proyectos similares para enseñar lenguajes de programación como \textit{Javascript} o \textit{Python}, como lo son \textit{CodingGames} o \textit{CodeCombat}, que de igual manera tienen entornos gráficos dinámicos y algunos están basados en géneros de juegos tradicionales como lo son los \textit{RPGs}.

Es importar notar que aprender a programar es una labor compleja donde se tienen que atacar distintos problemas al mismo tiempo. Programadores por primera vez tienen que aprender la sintaxis del lenguaje de programación de elección, así como sus idiosincrasias. Estos programadores a su vez tienen que aprender distintas abstracciones, como estructuras de datos para modelar problemas. Adicionalmente, tienen que aprender estrategias para la creación de algoritmos, como descomposición de problemas y reconocer patrones para buscar similitudes entre problemas.

En este trabajo se creó un videojuego multijugador. Todos los alumnos de una clase pueden entrar al mismo tiempo y jugar entre ellos. En este juego los estudiantes resuelven acertijos con pseudocódigo: tienen que analizar que tarea realiza un programa dadas ciertas entradas, completar programas para que realicen una tarea dada, de forma que trabajen su modelo mental sobre los distintos comandos vistos en el curso y trabajan en sus habilidades para crear algoritmos. Incluye los temas de: variables, estructuras secuenciales y ciclos, estos detallados en la retícula del curso de Fundamentos de Programación de la Universidad Autónoma de ciudad Juárez. En este videojuego, se divide a los jugadores en dos grupos: los programadores y los impostores. Los programadores, su principal objetivo es resolver todos los acertijos; los impostores tienen que impedir que los programadores terminen sus tareas, pueden hacerlo matando en juego a los programadores, o distrayéndolos, adicionalmente, pueden resolver acertijos para obtener extras que les de ventaja. 

En el primer capítulo de este trabajo se discuten trabajos previos del área, así como el planteamiento del problema y los objetivos del proyecto a realizar. A continuación, en el capítulo 2 se ve el marco conceptual y el marco tecnológico, en éste último explorando las herramientas a usar en la elaboración del producto. En el capítulo 3, se discute la metodología de desarrollo, se ve con más detalle el juego y el trabajo de fondo para su realización. El capítulo 4 trata los resultados de la verificación del producto, para medir que cumpla los temas del curso y sea algo que los profesores les interese usar en sus clases. Por último, en el capítulo 5, se discuten las conclusiones de la elaboración del proyecto.
