\documentclass[final, fmstyle, lic]{tesis}

% paquetes usados
%\usepackage[chapter, spanish]{theorems}
\newtheorem{theorem}{Teorema}[chapter]
\newtheorem{lemma}{Lema}[chapter]
\newtheorem{proposition}{Proposición}[chapter]
\newtheorem{corollary}{Corolario}[chapter]
\newtheorem{definition}{Definición}[chapter]
\newtheorem{example}{Ejemplo}[chapter]
\newtheorem{observation}{Observación}[chapter]

\usepackage{symbols}
\usepackage{graphicx} %usar imágenes externas
\usepackage{url}
\usepackage{blindtext}
\usepackage[table,xcdraw]{xcolor}

% Agregado por mi Saul para tener tablas que puedan usar varias paginas
\usepackage{longtable}
% Agregado por mi Saul para algunas citas directas
\usepackage{csquotes}
\usepackage{wrapfig}
\usepackage{float}

\usepackage{caption}
\usepackage{subcaption}
\DeclareCaptionSubType*[alph]{figure} % default is 'parens'
\captionsetup[subfigure]{labelformat=parens}
%\usepackage[labelformat=empty]{subcaption}
%\renewcommand\thesubfigure{Figura \thefigure.\alph{subfigure}:}

\usepackage[utf8]{inputenc}  % descomentar para teclado en español de Unix, Mac y posiblemente Windows
\usepackage[T1]{fontenc} % this helps hyphenate Spanish vocabulary with acute accent 
%\usepackage[latin1]{inputenc}  % descomentar para teclado en español de Unix y posiblemente también de Windows
%\usepackage[applemac]{inputenc}  % descomentar para teclado en español de Mac
\usepackage[spanish]{babel}
\decimalpoint

\setlength{\parskip}{2mm}	%Espaciado entre párrafos 

\graphicspath{ {./images/} } %Carpeta que contiene imagenes del documento

\usepackage{hyperref}
%\usepackage{urm}
% datos de la tesis
\title{Desarrollo de un videojuego multijugador para la enseñanza de la programación de computadoras}
\author{150424}{Saúl Núñez Castruita}
\authorb{}{}
\director{Mtro.}{Abraham López Nájera}
\directorb{Dr.}{Luis Carlos Méndez González}
% \directorbp{«Profesor-Investigador IIT»}  % Dejar en campo en blanco si no hay segundo asesor
\sinodala{Dr.}{Nombre del Sinodal A}
\sinodalb{Dr.}{Nombre del Sinodal B}
\sinodalc{Dr.}{Nombre del Sinodal C}
\titular{Dr.}{Francisco Lopez Orozco} % Titular de Seminario de Titulación II

% comenta para automático:
\termyear{\today}

\begin{document}
\pagenumbering{roman}

\maketitle     % esto hace las portadas
\approvalpage  % esto hace la pagina de aprobaci√õn
\approvalprint  % esto hace la pagina de aprobaci√õn
%\approvalprint % esto hace la pagina de aprobaci√õn de impresión. Comentar si sólo un alumno
\originalpage

\include{Agradecimientos}
\setcounter{page}{6}
% Inicio de Dedicatoria

\chapter*{Dedicatoria}


[Aquí escribe tu dedicatoria.]


% Fin de Dedicatoria

% los siguientes comandos producen Ìndices.
% tablas y figuras en el documento de la tesis
\tableofcontents
\listoffigures
\listoftables

% Inicio de Resumen
\chapter*{Resumen}
\markboth{chapter name}{Sinopsis}


% [Sustituye este texto escribiendo tu sinopsis o resumen. Es un panorama general de todo lo que el lector encontrará en tu documento, en no más de una página. Recuerda que este, junto con el título, son la parte más leída de tu documento cuando alguien más lo busca en las bases de datos, el “punto de venta''.]

En este trabajo se discute el proceso de la realización de un videojuego multijugador para el repaso de los temas vistos en la clase de Fundamentos de Programación, impartida en la Universidad Autónoma de Ciudad Juárez. 

Palabras claves: metodología de desarrollo, \textit{edutainment}, videojuegos educativos, enseñanza de programación de computadoras

% Fin del Resumen

\setcounter{page}{1}

\setcounter{page}{1}
\pagenumbering{arabic}
% Inicio de \section*{Introducción}\label{introduccion}
\chapter*{Introducción}\label{introduccion}

[Redacte de manera coherente en una cuartilla cuál es la nueva contribución, su importancia y por qué es adecuado para sistemas computacionales. Se sugiere para su redacción seguir los cinco pasos siguientes: 1) Establezca el campo de investigación al que pertenece el proyecto, 2) describa los aspectos del problema que ya han sido estudiado por otros investigadores, 3) explique el área de oportunidad que pretende cubrir el proyecto propuesto, 4) describa el producto obtenido y 5) proporcione el valor positivo de proyecto.]


\chapter{Planteamiento del Problema}
\section{Antecedentes}

\section{Definición del problema}

\section{objetivo general}

\subsection{Objetivos específicos}

\section{justificación}

\subsection{Alcances y limitaciones}
\chapter{Marco teórico}
En esta sección entramos en detalle el marco conceptual y el marco tecnológico. En la primera sección se verá la forma de aprovechar los videojuegos para la enseñanza, sus ventajas y la forma de enseñanza que estos proveen, además tocamos temas relacionados a la programación por bloques y la metodología de software a usar y sus artefactos. En la segunda sección discutimos de tecnologías aptas para la realización del proyecto.

\subsection{Marco Conceptual}
\subsubsection{Game-based learning y educación colaborativa}
Game-based learning es “en formación en la cual los contenidos teóricos son presentados por medio de un videojuego” \cite{gamelearn2014a} y contienen elementos como \cite{gamelearn2017a}:
\begin{itemize}
    \item Historia para darle inmersión a los jugadores
    \item Gamificación, como \textit{rankings} o un sistema de puntos
    \item \textit{Feedback} inmediato, algunos juegos tienen información como en que se equivocaron y una oportunidad de realizarlo otra vez
    \item Simulación de una situación de la vida real, permitiendo la práctica segura en ambientes cercanos a donde aplicarán su conocimiento
\end{itemize}

Los videojuegos pueden ser una muy poderosa herramienta para el aprendizaje: permiten el aprendizaje \textit{Just In Time} que bajo desafíos realizables empujan al 
jugador a ser competente y nos ayudan a fomentar el pensamiento crítico \cite{levasseur-a}. 
El aprendizaje \textit{Just In Time} ofrece el conocimiento necesario para hacer una tarea justo cuando es necesario \cite{unknown2017a}. 
Hay métodos en los que la enseñanza \textit{JIT} ocurre naturalmente como ver un video de \textit{Youtube} cuando no sabemos cómo realizar una tarea, 
en los videojuegos es natural cuando hay introducciones a acciones en el juego como la forma en la que lo invocamos o las acciones que realizan, 
ya sea mostrando una interfaz gráfica con la forma de invocarlo.
Se encontró que los estudiantes en el aprendizaje colaborativo los alumnos se enseñan uno al otro al responder a dudas y 
clarificar preconcepciones, desarrollan comunicación oral y capacidad de liderazgo, aumenta la retención del material, 
la responsabilidad y expone a los alumnos a perspectivas diversas \cite{university-a}.
Se ha encontrado que los videojuegos colaborativos agregan ventajas como el trabajo en equipo, el pensamiento creativo, 
la comunicación y la colaboración \cite{romano-a}.

\subsubsection{Kanban}
Kanban es una metodología ágil para el desarrollo de software con un énfasis 
en la entrega continúa teniendo en cuenta la capacidad del equipo \cite{romano-a}. 
Las métricas de Kanban son las siguientes \cite{najera2018a}:

\begin{figure}[h]
    \centering
    \includegraphics[width=0.5\textwidth]{flujo_acumulativo.png}
        \caption{Diagrama de flujo acumulativo}
    \label{fig:flujo_acumulativo}
\end{figure}

\begin{figure}[h]
    \centering
    \includegraphics{distribucion_tiempos_ciclos}
    \caption{Gráfica de distribución de tiempos de ciclo}
    \label{fig:distribucion_tiempos_ciclos}
\end{figure}

\begin{itemize}
    \item \textbf{Diagrama de flujo acumulativo:} Provee información relacionada con la capacidad del equipo. Está basado en tiempo y muestra cómo se mueven las tarjetas de izquierda a derecha en el tablero. Como se puede ver en la figura~\ref{fig:flujo_acumulativo} La altura de las bandas muestra el número de tarjetas en esa etapa durante cierta unidad de tiempo.
    \item \textbf{Gráfica de distribución de tiempos de ciclo:} Como se puede ver en la figura~\ref{fig:distribucion_tiempos_ciclos} es útil para ver la frecuencia con la que las tarjetas son completadas a lo largo del tiempo.
\end{itemize}

\subsection{Marco tecnológico}
Para el desarrollo de este proyecto es necesario tener herramientas para las partidas con otros usuarios, así como manera de usar librerías probadas y comúnmente usadas.

\subsubsection{Unity3D}
Es un motor que permite diseñar videojuegos. Permite usar C\# para programar el juego, compilando, usando Mono\cite{unity2019}. Es un motor muy capaz que permite desarrollar videojuegos 2D y 3D, con una variedad de herramientas para facilitar el desarrollo.
Unity puede crear el juego para varios entornos, como consolas de videojuegos como la Nintento Switch, el Xbox One, PS4, Windows, MacOS, Linux, Android, iOS, o la web mediante WebGL [24]. La ventaja del entorno web es la posibilidad de conectar Unity con la pagina web que lo alberga, con esto podemos conectarlo con librerías JS, como es el caso son la librería Blockly de Google de la cual hablaremos más adelante.
\begin{figure}[h]
    \centering
    \includegraphics{unity}
        \caption{Algunas capacidades de Unity para el desarrollo de videojuegos: animaciones, máquina de estados y para videojuegos 2D, un editor de sprites}
\end{figure}

\subsection{Mirror}
Mirror es una librería para crear juegos multijugador de Unity. Es una continuación a Unet de Unity Technologies. Permite desarrollar el cliente y el servidor en un único proyecto.

\subsubsection{Git}
Git es el sistema de control de versiones más popular del mundo, fue creado en 2005 por Linus Torvalds [27]. Es un ejemplo de un DVCS, un sistema distribuido de control de versiones, a diferencia de sistemas donde en un solo lugar esta todo el historial de versiones como CVS o Subversion. En Git todas las copias funcionales de código son un repositorio que contiene todo el historial de versiones.
En comparación con otros sistemas, Git está diseñado para que la creación de ramas y \textit{tags} sean operaciones baratas, por lo tanto, rápidas.

Github es un servicio de hospedaje de repositorios Git \cite{finley2012a}. Se uso como repositorio remoto para sincronizar cambios, además de aprovechar sus funciones para crear \textit{pipelines} para \textit{Continous Integration} permitiendo probar que el juego se pueda compilar y que ningún sistema este "roto".
\chapter{Desarrollo del Proyecto}
[Este capítulo se considera el más importante al elaborar el proyecto de titulación. Se describe el procedimiento seguido para lograr el objetivo planteado. Se explica qué y cómo se hizo, además se debe de convencer de que los métodos o procedimientos usados fueron los más adecuados.

Deben detallarse los procedimientos, técnicas, métodos, metodologías y demás estrategias metodológicas requeridas para el proyecto. 
]

\section{Producto propuesto}

\section{Fases (Metodología)}



...
\section{Avances}



\chapter{Resultados y Discusiones}
En este capítulo analizaremos las pruebas que se realizaron al software para medir su efectividad. Se revisarán y analizarán los resultados de las pruebas cuantitativas y cualitativas del producto realizado.

\subsection{Resultados}
Se realizaron dos pruebas. Una prueba con N personas, con NH siendo hombres y NM mujeres, esta prueba se llevó a cabo el FECHA_PRUEBA, esta consistió en juntar a varias personas en una llamada por internet, para simular el entorno de un salón de clase, se les paso la información del lobby, se le dio instrucciones de la forma de juego a los participantes y se empezó el juego. La segunda prueba con M personas, con MH siendo hombres y MM mujeres, esta prueba ocurrió el FECHA_PRUEBA2, esta prueba se realizó de igual manera mediante una llamada por internet, fueron revisados diversos temas de programación estructural, detallados en el ANEXO_MATERIAL, donde se realizaron diversos ejercicios y al completar se les pidió a los participantes resolver un examen detallado en ANEXO_EXAMEN.
El promedio de los resultados de la primera prueba fueron P1, con una media de M1. En el caso de la segunda prueba el promedio fue P2, con una media de M2. 
La distribución de los resultados de la primera prueba se puede apreciar en la Ilustración 16 y los de la segunda prueba, la cual es de _ es la de la Ilustración 17, entre estas dos podemos notar X_CAMBIOS_DESCRIPCION.
Usando la prueba de la U de Mann-Whitney podemos encontrar que en la prueba X fue la que obtuvo un mejor rendimiento.
A ambos grupos además se les hizo una pequeña encuesta sobre cómo les pareció la clase, si se les hizo entretenida, como evalúan su capacidad sobre lo aprendido, la evaluación se puede ver en el ANEXO_ENCUESTA. .{..ANALISIS_DE_ENCUESTA}

\subsection{Discusiones}
Las discusiones por lo regular comienzan con unos cuantos enunciados que sumarizan los resultados más importantes para introducirnos en la discusión. Las discusiones deben ser breves y responder a las siguientes preguntas:
¿Cuáles son los patrones más importantes que observamos?
¿Cuáles son las relaciones, tendencias y generalizaciones entre los resultados?
¿Cuáles son las excepciones o generalizaciones a esos patrones?
¿Cuáles son las causas más probables?
¿Cuáles son las causas más probables de los patrones resultantes?
¿Hay acuerdo o desacuerdo con trabajos previos?

La discusión puede mencionar someramente los resultados antes de discutirlos, pero no debe repetirlos en detalle. No prolongues la discusión citando trabajos "relacionados" o planteando explicaciones poco probables. Ambas acciones distraen al lector y lo alejan de la discusión realmente importante. La discusión puede incluir recomendaciones y sugerencias para investigaciones futuras, tales como métodos alternos que podrían dar mejores resultados, tareas que no se hicieron y que en retrospectiva debieron hacerse, y aspectos que merecen explorarse en las próximas investigaciones. 

% Inicio de Introducción
\chapter{Conclusiones}\label{conclusiones}
 
 En esta sección se resume el trabajo realizado. Adicionalmente, se discute los resultados respecto a los objetivos y la pregunta de investigación. Adicionalmente se discute de trabajos a futuro.
 
\section{Con respecto a las preguntas de investigación}

Se encontró interés en muchos docentes de la materia de Fundamentos, respecto a las ventajas que pudiera traer un juego como este, así como de las capacidades actuales del producto. Uno de los docentes entrevistados menciono que lo usaría para su clase cuando llegue a lanzarse. 

En general, todos los docentes estuvieron de acuerdo en que sería algo que llamaría la atención a sus estudiantes, entre las razones porque:
\begin{itemize}
    \item Los roles. El cambio constante de los roles, así como la complejidad adicional que agrega la interacción entre compañeros para ser efectivo en tu rol mantienen el juego interesante para los alumnos.
    \item A base de su experiencia, los alumnos responden bien a cuando usan juegos o aplicaciones para el proceso de enseñanza. Juegos como \textit{Kahoot!} tuvieron muy buena acogida  en el salón de clases.
\end{itemize}

\section{Con respecto al objetivo de la investigación}
\begin{itemize}
    \item Se realizó un documento de diseño donde se diseñaron las mecánicas del juego, ambientación, personajes y \textit{puzzles}.
    \item Se creó un juego a base de este documento usando Unity3D y Mirror Netorking para el multijugador; este juego está listo para albergar partidas del juego, puede ser instalado en un servidor o en una computadora en una red local. Estos ejecutables esta contenerizados y son orquestados usando Docker Compose.
    \item Se realizaron reuniones con profesores mostrándoles el juego y tuvieron interés en el juego. Descubriendo \textit{showstoppers}, los cuales se corrigieron, de forma que el juego cumple de mejor manera los requerimientos que los docentes necesitan para usarlo en sus clases.
\end{itemize}

\section{Recomendaciones para futuras investigaciones}
\subsection{Puzzles}
Hubo dos sugerencias, una sobre agregar temas adicionales y otra sobre la complejidad de los ejercicios (que esta pudiera subir más). Para cubrir esta área, se pueden agregar lo siguientes al juego como trabajo futuro:
\begin{itemize}
    \item Crear ejercicios con los temas de arreglos y matrices.
    
    \item Agregar ejercicios más complejos. Los ejercicios de juego fueron diseñados para ser sencillos: de un solo tema y no muy extensos. Se puede a futuro aumentar la complejidad con la creación de ejercicios con más condicionales dobles, y agregar \texit{puzzles} con condicionales anidadas, ciclos y combinación de ciclos con condicionales. Estos ejercicios pudieran ser habilitados o deshabilitados en Configuración de partida y si están habilitados que reemplacen algunos de los ejercicios existentes de su respectivo tema. De forma que los alumnos tengan que razonar más el ejercicio. Y como efecto secundario, esto a su vez mantiene el juego fresco durante más tiempo. En general esto permite al docente ajustar la dificultad según el nivel de su clase.
    
    \item Usando la Configuración de partida, se pudiera agregar una nueva opción que solo permita \textit{puzzles} que tengan contenido de programación, dado que algunos ejercicios para las emergencias tienen preguntas de conocimiento popular.
\end{itemize}

\subsection{Pruebas Unitarias}
Aunque existe la infraestructura en el sistema de integración continua para ejecutar pruebas unitarias al integrar nuevo código a la rama \texit{master}. Falta mucho para mejorar la cobertura del código con pruebas de este tipo. 

\subsection{Comprobación con alumnos}
Uno de los temas comunes de todas las reuniones con docentes es que un juego como ayuda pedagógica es algo que sería de interés para los alumnos. Sería interesante realizar una encuesta para obtener la opinión de los estudiantes de educación superior sobre aprender con juegos, esto permitiría evaluar la utilidad de creación de diversos proyectos de juegos para \textit{e-learning} para este segmento. 

Adicionalmente seria de especial interés realizar una verificación con estudiantes. En esta se podría estudiar si hay un impacto positivo en el desempeño de los estudiantes con el uso del material de repaso aquí creado. Así como encontrar detalles que impacten la usabilidad del programa y evaluar si a los alumnos se les hace divertido repasar con el juego. Porque, aunque es vital aprender que opinión tienen los profesores sobre el juego porque son ellos quien lo implementaran en el salón de clases, ya en el salón de clases es importante ver su impacto y la forma en la que podemos mejorar la utilidad de este producto.



% Fin de Conclusiones


\bibliographystyle{ieeepes}
\bibliography{referencias}


\appendix   % inician los apÈndices de la tesis

% los capÌtulos que se incluyan a partir de aquÌ aparecen 
% como apÈndices
% Inicio del ApÈndice A
\chapter{Nombre del Apéndice}\label{apendiceA}

\section{Imágenes}
\begin{figure}[H]
\centering
         \begin{subfigure}{0.8\textwidth}
         \centering
         \includegraphics[width=\textwidth]{images/PreguntaEncuesta (4).png}
         \caption{Pregunta 1.}
         \label{fig:survey1}
    \end{subfigure}

        \begin{subfigure}{0.8\textwidth}
         \centering
         \includegraphics[width=\textwidth]{images/PreguntaEncuesta (5).png}
         \caption{Pregunta 2.}
         \label{fig:survery2}
     \end{subfigure}
\end{figure}
\begin{figure}
\ContinuedFloat
    \centering
    \begin{subfigure}{0.8\textwidth}
         \centering
         \includegraphics[width=\textwidth]{images/PreguntaEncuesta (1).png}
         \caption{Pregunta 3.}
         \label{fig:survey3}
    \end{subfigure}
    \begin{subfigure}{0.8\textwidth}
         \centering
         \includegraphics[width=\textwidth]{images/PreguntaEncuesta (2).png}
         \caption{Pregunta 4.}
         \label{fig:survery4}
     \end{subfigure}
    \begin{subfigure}{0.8\textwidth}
         \centering
         \includegraphics[width=\textwidth]{images/PreguntaEncuesta (3).png}
         \caption{Pregunta 5.}
         \label{fig:survery5}
    \end{subfigure}
        \caption{Preguntas de la encuesta a docentes respecto al juego.}
    \label{fig:survey}
\end{figure}

\section{Tablas}
\begin{longtable}[c]{ >{\centering\arraybackslash}p{2.75cm} >{\centering\arraybackslash}p{2.25cm} >{\centering\arraybackslash}p{2.25cm} >{\centering\arraybackslash}p{2.25cm} >{\centering\arraybackslash}p{2.25cm} >{\centering\arraybackslash}p{2.25cm}}
\caption{Resultados obtenidos de la encuesta a profesores\label{table:resultados_survey}.} \\
\linespread{0.5}\selectfont\centering
\setlength{\tabcolsep}{2pt}
\renewcommand{\arraystretch}{0.75}
         & Docente 1 & Docente 2 & Docente 3 & Docente 4 & Docente 5 \\ \hline
        Temas tratados en los puzzles & Neutral & Satisfecho & Satisfecho & Satisfecho & Satisfecho \\ \hline
        Complejidad de los puzzles & Neutral & Satisfecho & Satisfecho & Satisfecho & Satisfecho \\ \hline
        Número de puzzles & Neutral & Satisfecho & Neutral & Neutral & Neutral \\ \hline
        Mecánicas del juego & Insatisfecho & Satisfecho & Satisfecho & Muy satisfecho & Satisfecho \\ \hline
        Roles de juego cambiantes & Neutral & Neutral & Satisfecho & Muy satisfecho & Satisfecho \\ \hline
        Los estudiantes pueden aprender juntos & Muy satisfecho & Satisfecho & Satisfecho & Satisfecho & Muy satisfecho \\ \hline
        ¿Qué temas le gustaría que tratara adicionalmente el juego? & Temas acordes a la carta descriptiva & Arreglos y matrices & Arreglos y matrices. & Arreglos, matrices, código C básico. & Aumentar la complejidad, anexar arreglos y SubProcesos \\ \hline
        ¿Cuántos estudiantes tiene que soportar el juego al mismo tiempo para que el juego se útil en sus clases? & 30 & Sería bueno 30  & 30 & 20 mínimo & Mis grupos regularmente son de 40 estudiantes \\ \hline
        ¿El método de instalación donde se tiene que descargar el juego y ejecutarlo usando Docker es óptimo para usarlo en el salón de clases? & 7 & 10 & 7 & 8 & 9 \\ \hline
        ¿Teniendo en consideración la demostración del producto, cual es la probabilidad de que lo use en el salón de clases? & 5 & 8 & 7 & 10 & 9 \\ \hline
\end{longtable}

% estos comandos generan la bilbiografÌa
% La bibliografÌa se obtiene de la base de datos
% Estilos:
%	 plain (sistema numÈrico, orden alfabÈtico)
%	 unsrt (Sistema numÈrico, en el orden en que van apareciendo las citas) ------
%	 alpha (Sistema autor-fecha abreviado, orden alfabÈtico)
%	 abbrv (Sistema autor-fecha abreviado, estilo bibliogr·fico alfabÈtico abreviado)
%	 ieeepes (Bibliography Style file for articles according to IEEE instructions) Basado en unsrt



\end{document}