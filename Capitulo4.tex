\chapter{Resultados y Discusiones}
En este capítulo analizaremos las pruebas que se realizaron al software para medir su efectividad. Se revisarán y analizarán los resultados de las pruebas cuantitativas y cualitativas del producto realizado.

\subsection{Resultados}
Se realizaron dos pruebas. Una prueba con N personas, con NH siendo hombres y NM mujeres, esta prueba se llevó a cabo el FECHA_PRUEBA, esta consistió en juntar a varias personas en una llamada por internet, para simular el entorno de un salón de clase, se les paso la información del lobby, se le dio instrucciones de la forma de juego a los participantes y se empezó el juego. La segunda prueba con M personas, con MH siendo hombres y MM mujeres, esta prueba ocurrió el FECHA_PRUEBA2, esta prueba se realizó de igual manera mediante una llamada por internet, fueron revisados diversos temas de programación estructural, detallados en el ANEXO_MATERIAL, donde se realizaron diversos ejercicios y al completar se les pidió a los participantes resolver un examen detallado en ANEXO_EXAMEN.
El promedio de los resultados de la primera prueba fueron P1, con una media de M1. En el caso de la segunda prueba el promedio fue P2, con una media de M2. 
La distribución de los resultados de la primera prueba se puede apreciar en la Ilustración 16 y los de la segunda prueba, la cual es de _ es la de la Ilustración 17, entre estas dos podemos notar X_CAMBIOS_DESCRIPCION.
Usando la prueba de la U de Mann-Whitney podemos encontrar que en la prueba X fue la que obtuvo un mejor rendimiento.
A ambos grupos además se les hizo una pequeña encuesta sobre cómo les pareció la clase, si se les hizo entretenida, como evalúan su capacidad sobre lo aprendido, la evaluación se puede ver en el ANEXO_ENCUESTA. .{..ANALISIS_DE_ENCUESTA}

\subsection{Discusiones}
Las discusiones por lo regular comienzan con unos cuantos enunciados que sumarizan los resultados más importantes para introducirnos en la discusión. Las discusiones deben ser breves y responder a las siguientes preguntas:
¿Cuáles son los patrones más importantes que observamos?
¿Cuáles son las relaciones, tendencias y generalizaciones entre los resultados?
¿Cuáles son las excepciones o generalizaciones a esos patrones?
¿Cuáles son las causas más probables?
¿Cuáles son las causas más probables de los patrones resultantes?
¿Hay acuerdo o desacuerdo con trabajos previos?

La discusión puede mencionar someramente los resultados antes de discutirlos, pero no debe repetirlos en detalle. No prolongues la discusión citando trabajos "relacionados" o planteando explicaciones poco probables. Ambas acciones distraen al lector y lo alejan de la discusión realmente importante. La discusión puede incluir recomendaciones y sugerencias para investigaciones futuras, tales como métodos alternos que podrían dar mejores resultados, tareas que no se hicieron y que en retrospectiva debieron hacerse, y aspectos que merecen explorarse en las próximas investigaciones. 
